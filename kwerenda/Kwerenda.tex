\documentclass{article}

% ---- POLSKIE ZNAKI ----
\usepackage[utf8]{inputenc}
\usepackage[T1]{fontenc}
\usepackage[polish]{babel}

% ---- UKŁAD I TABELE ----
\usepackage[margin=1in]{geometry}
\usepackage{ltablex}
\usepackage{array}
\usepackage{xurl}
\usepackage{caption}
\usepackage{pdflscape}
\usepackage{xcolor}

% ---- USTAWIENIA ----
\renewcommand\tabularxcolumn[1]{>{\raggedright\arraybackslash}p{#1}}
\setlength{\extrarowheight}{2pt}
\pagestyle{empty} % brak numerów stron

\begin{document}

\begin{landscape} % tryb poziomy

\begin{small}
\keepXColumns

% ----------------- TABELA -----------------
\begin{tabularx}{\linewidth}{|p{0.4cm}|p{4cm}|p{3.4cm}|p{4cm}|X|}
\hline
\textbf{Lp} &
\textbf{Tytuł} &
\textbf{Autorzy} &
\textbf{DOI} &
\textbf{Abstrakt + wnioski} \\
\hline
\endfirsthead


1.&Engagement in In-Game Questionnaires - Perspectives from Users and Experts &
Alexander Steinmaurer, Martin Sackl, Christian Gütl &
\url{https://ieeexplore.ieee.org/document/9459373} &
Kwestionariusze są istotnym narzędziem pozyskiwania informacji od badanych i cechują się szerokim zakresem zastosowań, takich jak analiza odpowiedzi w celach ewaluacyjnych, zbieranie preferencji czy samoocena w kontekstach edukacyjnych. Jednak w zależności od sposobu zaprojektowania i umiejscowienia kwestionariusza, uczestnictwo w badaniu może być dla respondentów nużące lub frustrujące, co negatywnie wpływa na wyniki lub prowadzi do utraty zaangażowania. Literatura wskazuje na \textbf{negatywne efekty w symulacjach immersyjnych oraz grach edukacyjnych, w których kwestionariusze i oceny są realizowane poza właściwym doświadczeniem edukacyjnym.}

W niniejszym artykule przedstawiamy podejście polegające na zintegrowaniu kwestionariuszy z otoczeniem gry. Naszym celem było \textbf{stworzenie angażującej formy odpowiadania na pytania przez uczestników}. Aby dodatkowo ich zmotywować, za wypełnianie kwestionariuszy przewidziano system nagród. Przeprowadziliśmy dwie ewaluacje: badanie A/B z udziałem 22 uczestników oraz ocenę z udziałem 14 ekspertów z dziedzin związanych z tematyką gry. Wyniki wykazały, że \textcolor{red!70!black}{uczestnicy byli bardziej zaangażowani w wypełnianie kwestionariuszy osadzonych w grze, a sama integracja pytań zwiększyła skłonność do udzielania odpowiedzi.}

Dodatkowo zidentyfikowano strategie pozwalające uzyskać bardziej wiarygodne odpowiedzi, takie jak zadawanie pytań bezpośrednio po wykonaniu zadania lub wykorzystanie elementów grywalizacji. Wyniki mogą przyczynić się do projektowania bardziej angażujących aplikacji i środowisk edukacyjnych, w których ewaluacja i informacje zwrotne mają znaczenie zarówno dla nauczycieli, jak i dla uczniów.

 \\
\hline

2. &Trading Accuracy for Enjoyment? Data Quality and Player Experience in Data Collection Games &
David J. Gundry, Sebastian Deterding &
\url{https://dl.acm.org/doi/10.1145/3491102.3502025} &
Gry stały się popularnym sposobem zbierania danych od uczestników badań, opartym na założeniu, że są one bardziej angażujące niż tradycyjne ankiety czy eksperymenty, a jednocześnie dostarczają równie wartościowych danych. Jednak to założenie nie zostało dotychczas poddane empirycznej weryfikacji. W odpowiedzi na tę lukę \textbf{zaprojektowaliśmy grę do zbierania danych językowych w oparciu o podejście Intrinsic Elicitation – metodę projektowania mającą na celu minimalizowanie zagrożeń dla trafności w grach służących do zbierania danych – i porównaliśmy ją z równoważnym eksperymentem językoznawczym pełniącym rolę grupy kontrolnej.}

W prerejestrowanym badaniu i jego replikacji (n=96 i n=136), wykorzystując dwa różne sposoby operacjonalizacji dokładności, gra zapewniała znacznie większą przyjemność z udziału (d=0.70, 0.73), ale jednocześnie generowała istotnie mniej dokładne dane (d=-0.68, -0.40) — choć nadal bardziej trafne niż odpowiedzi losowe. Wnioskujemy, że \textcolor{red!70!black}{w przypadku niektórych typów danych gry do zbierania danych mogą wiązać się z poważnym kompromisem pomiędzy przyjemnością uczestników a jakością danych}. Identyfikujemy możliwe przyczyny niższej jakości danych do dalszych badań, dokonujemy refleksji nad naszym podejściem projektowym oraz apelujemy do badaczy HCI zajmujących się grami o stosowanie odpowiednich grup kontrolnych tam, gdzie jest to uzasadnione. \\
\hline

3. &Gathering Self-Report Data in Games Through NPC Dialogues: Effects on Data Quality, Data Quantity, Player Experience, and Information Intimacy &
Julian Frommel, Cody Phillips, Regan L. Mandryk &
\url{https://dl.acm.org/doi/10.1145/3411764.3445411} &
Ocena za pomocą samoopisu odgrywa istotną rolę zarówno w badaniach naukowych, jak i w projektowaniu gier, np. w celu gromadzenia danych w trakcie rozgrywki. Gry mogą wykorzystywać dialogi z postaciami niezależnymi (NPC) do zbierania danych samoopisowych, jednak gracze mogą reagować na nie inaczej niż na tradycyjne kwestionariusze. Bez jasnych wskazówek dotyczących wpływu oceny osadzonej w grze na postrzeganie i doświadczenia gracza, projektanci i badacze ryzykują podejmowanie decyzji, które mogą negatywnie wpłynąć na ilość i jakość danych oraz na poczucie prywatności uczestników.

Przeprowadziliśmy \textbf{badanie użytkowników, aby porównać zbieranie danych samoopisowych za pomocą dialogów NPC i tradycyjnych kwestionariuszy w formie nakładek w grze.} Jakość danych oraz doświadczenie gracza mierzone poprzez autonomię, ciekawość, immersję i poczucie mistrzostwa nie różniły się istotnie pomiędzy formami, choć dialogi NPC nadawały większe znaczenie kontekstowi.

\textcolor{red!70!black}{Dialogi NPC przyczyniły się do zwiększenia ilości danych przy zastosowaniu dobrowolnych skal pięciopunktowych, lecz nie w przypadku pytań otwartych.} Jednocześnie zwiększały postrzeganą intymność przekazywanych informacji, mimo że ich rzeczywista intymność była porównywalna do danych uzyskanych z kwestionariuszy.

Dialogi NPC mogą być przydatne do zbierania ilościowych danych samoopisowych, sprzyjając jednocześnie bardziej znaczącemu doświadczeniu gry. Należy jednak uwzględnić możliwość wystąpienia negatywnych efektów związanych z postrzeganiem prywatności. \\
\hline



4. &Advancing Psychological Research: A Controlled Virtual Reality Environment for Exploring Social Dynamics in Immersive Conversational Interactions &
Tung Khau, Giuseppe De Luca, Martina Benvenuti Elvis Mazzoni, Gerrit Meixner &
\url{https://ieeexplore.ieee.org/document/10628556} &
Artykuł przedstawia innowacyjne opracowanie kontrolowanego środowiska wirtualnej rzeczywistości (VR) zaprojektowanego na potrzeby eksperymentów psychologicznych i społecznych, wykorzystującego znany paradygmat Cyberball jako punkt wyjścia. Przejście od tradycyjnych, fizycznych wersji Cyberball do immersyjnego środowiska VR stanowi istotny krok naprzód w zakresie kontroli eksperymentalnej i realizmu. W przeciwieństwie jednak do konwencjonalnych implementacji Cyberball, nasze podejście odchodzi od klasycznej aktywności polegającej na rzucaniu piłki, koncentrując się na \textbf{badaniu dynamiki interpersonalnej za pomocą interakcji konwersacyjnych w wirtualnym pokoju rozmów w rzeczywistości wirtualnej.}

W pracy szczegółowo omówiono projektowanie, implementację i ewaluację tego kontrolowanego środowiska VR, podkreślając jego kluczową rolę w kształtowaniu przyszłości badań psychologicznych. Uwzględniono w nim m.in. nagrane kwestie dialogowe, animacje wirtualnych postaci oraz strategie zwiększania zaangażowania uczestników. Szczegółowa analiza paradygmatu Cyberball w ramach tego środowiska VR przyczynia się do \textcolor{red!70!black}{opracowania dobrych praktyk i wytycznych w projektowaniu kontrolowanych środowisk wirtualnych, sprzyjając lepszemu zrozumieniu dynamiki społecznej i zjawisk psychologicznych.} \\
\hline

5. &Digitalising qualitative social research? On the potential of digital features to enhance data collection in qualitative research: the example of a virtual reality serious game in a qualitative research &
Daniel Gredig, Daniele Bigoni, Jasmina Bogdanovic, Patrick Weber and Safak Korkut&
\url{https://bristoluniversitypressdigital.com/view/journals/eswr/1/1/article-p118.xml} &
Cyfrowe rozwiązania, takie jak wirtualna rzeczywistość (VR), rzadko były wykorzystywane w ramach zbierania danych w jakościowych badaniach w obszarze pracy socjalnej. VR wydaje się jednak szczególnie obiecującym narzędziem, ponieważ \textbf{pozwala uczestnikom zanurzyć się w sytuacji badawczej i może zwiększać trafność ekologiczną zbieranych danych.} W tym kontekście przeanalizowaliśmy wykorzystanie poważnej gry VR (serious game) w jakościowych wywiadach bezpośrednich, przeprowadzanych w ramach projektu badawczego ukierunkowanego na prewencję HIV.

Zaprojektowaliśmy i opracowaliśmy immersyjną grę VR, którą następnie zintegrowaliśmy z 24 wywiadami problemowo-centrycznymi. Integracja okazała się wykonalna, a gra została dobrze przyjęta przez uczestników. Zachęcała ich do \textcolor{red!70!black}{bardziej rozbudowanego i pogłębionego opowiadania o własnych doświadczeniach}. Uczestnicy przypominali sobie epizody, o których wcześniej nie wspomnieli, uzupełniali swoje narracje i poruszali nowe tematy. Atmosfera wywiadów stawała się mniej formalna, a badani byli bardziej otwarci i komunikatywni.

\textcolor{red!70!black}{Zastosowanie VR wydaje się otwierać nowe perspektywy, poszerzać możliwości epistemiczne i wzbogacać jakościowe metody badawcze, szczególnie w zakresie analizy komunikacji werbalnej i niewerbalnej oraz procesów interakcji}. Z perspektywy etycznej wykorzystanie VR wymaga jednak starannego rozważenia potencjalnych skutków ubocznych dla uczestników badania. \\
\hline


6. &Role-Playing Games as an Assessment Strategy for Social Skills &
Kate A. Helbig, Evan H. Dart, Keith C. Radley &
\url{https://www.igi-global.com/gateway/chapter/333591} &
W tym rozdziale omówiono koncepcję ukrytej oceny (stealth assessment) jako narzędzia do oceny komunikacji społecznej. Dotychczas sceny odgrywania ról w ramach stealth assessment opierały się na zaplanowanych i scenariuszowych aktywnościach, których celem było wywołanie określonych reakcji społecznych. Reakcje te składały się zazwyczaj z wyraźnie zdefiniowanych kroków ocenianych za pomocą bezpośredniej obserwacji. Takie podejście było jednak krytykowane za sztuczność i sztywność, co ograniczało możliwości oceny uogólnionych umiejętności w bardziej naturalistycznych warunkach.

Gry fabularne (RPG) stanowią elastyczne narzędzie oceny, które \textbf{sprzyja zaangażowaniu i motywacji dzięki formie opartej na grze. Udział badanych w mniej sztywnym scenariuszu gry pozwala oceniającym aranżować nieskończoną liczbę interakcji i sytuacji}, umożliwiając pozyskanie różnorodnych przykładów zachowań społecznych i tym samym prowadząc do bardziej naturalistycznej oceny umiejętności społecznych uczestników.

Celem tego rozdziału jest przedstawienie wykorzystania ukrytej oceny zachowań społecznych w kontekście RPG, omówienie strategii dla oceniających oraz przegląd wyników empirycznej oceny podejścia RPG jako narzędzia diagnostycznego. \\
\hline

7. &The effects of role-playing high gender identification on the California psychological inventory &
James F. Montross, Faye Neas, Christina L. Smith, John H. Hensley &
\url{https://doi.org/10.1002/1097-4679(198803)44:2<160::AID-JCLP2270440211>3.0.CO;2-W} &
Badanie dotyczące California Psychological Inventory (CPI) pokazuje, że uczestnicy mogą znacząco zmieniać swoje profile CPI poprzez odgrywanie ról z silną identyfikacją płciową, tzn. \textcolor{red!70!black}{mężczyźni starają się wypaść jako bardziej męscy, a kobiety jako bardziej kobiece}. Większość badanych była w stanie to osiągnąć bez wykrycia przez skale trafności, które służą do identyfikowania prób fałszowania wyników w pozytywnym lub negatywnym kierunku.

W badaniu wzięło udział 32 ochotników — 16 mężczyzn i 16 kobiet. Zastosowano schemat analizy wariancji 2 × 2 (płeć uczestnika × instrukcja odgrywania roli vs. instrukcja standardowa). Stwierdzono istotny główny efekt (p < 0,01) dla instrukcji na 14 z 18 skal oraz istotną interakcję (p < 0,01) na 4 z 18 skal. Profile osób odgrywających role wskazują, że stereotypy obu płci — tak jak postrzegają je uczestnicy — mają charakter nieadaptacyjny.

W artykule omówiono również praktyczne implikacje uzyskanych wyników. \\
\hline


8. &Effects of scene manipulation on role-play test behavior &
Robert T. Ammerman, Michel Hersen &
\url{https://doi.org/10.1007/BF00960873} &
Wiele badań wykazało, że manipulowanie elementami procedury testu odgrywania ról różnicuje zachowanie uczestników. Obszarem, który nie był jednak szeroko badany w kontekście paradygmatu role-play, jest \textbf{wpływ oczekiwań dotyczących wyniku (outcome expectancy) na zachowania interpersonalne.} W niniejszym badaniu oceniono zależność między oczekiwaniami a zachowaniem, manipulując opisami scen w teście role-play. Do opisów dodano informacje sugerujące, że osoba odgrywająca rolę partnera interakcji jest łatwa do nawiązania kontaktu (pozytywne oczekiwania) lub trudna do nawiązania kontaktu (negatywne oczekiwania).

Następnie interakcje odgrywane przez uczestników zostały ocenione na podstawie różnych wskaźników werbalnych i niewerbalnych kompetencji społecznych. Dane wykazały, że \textcolor{red!70!black}{uczestnicy byli mniej skuteczni społecznie w scenach z negatywnymi oczekiwaniami. Z kolei w scenach z pozytywnymi oczekiwaniami ich zachowania były oceniane jako nieco bardziej efektywne społecznie. Dodatkowo osoby o wysokim poziomie lęku (określonym na podstawie samoopisowych kwestionariuszy) okazały się bardziej podatne na manipulację oczekiwaniami niż osoby o niskim poziomie lęku.}

Wyniki omówiono w kontekście (1) roli oczekiwań dotyczących wyniku w zachowaniach interpersonalnych oraz (2) potrzeby dalszego udoskonalania testów typu role-play. \\
\hline

\end{tabularx}

\end{small}

\end{landscape} % <--- koniec strony poziomej

\end{document}